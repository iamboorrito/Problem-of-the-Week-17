\documentclass[]{article}
\usepackage{amsfonts}

%opening
\title{Problem 9}
\author{Evan Burton, ID: 010945129, Undergraduate}

\begin{document}

\maketitle

\section{Problem}
Let $x_0 = 0$, $x_1 = 1$, and for $n \geq 1$,

$$x_{n+1} =\frac{x_n}{n+1} + (1 - \frac{1}{n+1})x_{n-1}$$  

Determine $\displaystyle{\lim_{n \rightarrow \infty}} x_n$.
   
\section{Solution}
Suppose $n \in \mathbb{N}, n \geq 1$, then $x_n = \displaystyle{\sum_{k=1}^{n} \frac{(-1)^{k+1}}{k}}$

For $n = 1$:

$x_1 = \displaystyle{\sum_{k=1}^{1} \frac{(-1)^{k+1}}{k}} = \frac{1}{1} = 1$\\

Suppose the proposition is true up to $n$ and $n > 1$, we need to show that the proposition is still true for $n+1$.

$$x_{n+1} = \displaystyle{\sum_{k=1}^{n+1} \frac{(-1)^{k+1}}{k}} = 
\frac{(-1)^{n+2}}{n+1} + \displaystyle{\sum_{k=1}^{n} \frac{(-1)^{k+1}}{k}}
$$

$$=
\frac{(-1)^{n+2}}{n+1} + \frac{(-1)^{n+1}}{n} + \displaystyle{\sum_{k=1}^{n-1} \frac{(-1)^{k+1}}{k}}
$$

$$=
(-1)^{n+1}\left[\frac{1}{n} - \frac{1}{n+1}\right] + \displaystyle{\sum_{k=1}^{n-1} \frac{(-1)^{k+1}}{k}}
$$

$$=
\frac{(-1)^{n+1}}{n(n+1)}+\displaystyle{\sum_{k=1}^{n-1} \frac{(-1)^{k+1}}{k}}
=
\frac{1}{n+1}\left[\frac{(-1)^{n+1}}{n}+{(n+1)}\displaystyle{\sum_{k=1}^{n-1} \frac{(-1)^{k+1}}{k}}\right]
$$

$$=
\frac{1}{n+1}\left[\frac{(-1)^{n+1}}{n}+n\displaystyle{\sum_{k=1}^{n-1} \frac{(-1)^{k+1}}{k}}+\displaystyle{\sum_{k=1}^{n-1} \frac{(-1)^{k+1}}{k}}\right]
$$

$$=
\frac{1}{n+1}\left[n\displaystyle{\sum_{k=1}^{n-1} \frac{(-1)^{k+1}}{k}}+\displaystyle{\sum_{k=1}^{n} \frac{(-1)^{k+1}}{k}}\right] = \frac{nx_{n-1} +x_n}{n+1}
$$

$$=
\frac{x_n}{n+1} + \frac{nx_{n-1}}{n+1}
=
\frac{x_n}{n+1} + \left(1 - \frac{1}{n+1}\right)x_{n-1}
$$


Which is the definition of $x_{n+1}$ stated in the problem.\\

Therefore, $x_n = \displaystyle{\sum_{k=1}^{n} \frac{(-1)^{k+1}}{k}}$ and the limit 

$$\displaystyle{\lim_{n\rightarrow \infty} x_n} = \lim_{n\rightarrow \infty} \displaystyle{\sum_{k=1}^{n} \frac{(-1)^{k+1}}{k}} = ln(2)$$

Of course, a proof by induction is not very constructive. My motivation for the proposition was that the sequence ${x_n-x_{n-1}}$ gave the terms of the Maclaurin series of $ln(1+x)$ at $x = 1$ and the first few values of $x_n$ seemed to confirm this, so it was a natural candidate for induction. 
\end{document}
