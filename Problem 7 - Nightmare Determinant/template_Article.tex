\documentclass[]{article}
\usepackage{amsmath}
%opening
\title{Problem 7}
\author{Evan Burton, ID:010945129, Undergraduate}

\begin{document}

\maketitle

\section{Problem Statement}
Show that the determinant of A is an integer, where \\

$$ A = \left[
\begin{matrix}
cos(1) & cos(11) & \cdots & cos(91) \\ 
cos (2) & cos (12) & \cdots & cos (92) \\ 
\vdots & \vdots  & \ddots & \vdots \\ 
cos (10) & cos (20) & \cdots & cos (100)
\end{matrix}\right]
$$
 and the angles are in radians.
\section{Solution}
Consider $e^{i\theta} = cos(\theta) + i sin(\theta)$. Since cosine is even, $cos(-\theta) = cos(\theta)$ and since sine is odd, $sin(-\theta) = -sin(\theta)$. \\

Then, $e^{-i \theta} = e^{i(-\theta)} = cos(-\theta) + i sin(-\theta) = cos(\theta) - i sin(\theta)$.

$$cos(\theta) = \frac{1}{2}(e^{i\theta} + e^{-i\theta}) = cosh(i\theta)$$



We will need a property of hyperbolic cosine:
$$4 cosh(i\theta)cosh(i) = (e^{i\theta} + e^{-i\theta})(e^i + e^{-i})$$
$$= e^{i\theta + i} + e^{i\theta - i} + e^{-i\theta + i} + e^{-i\theta - i}$$
$$= e^{i(\theta + 1)} + e^{i(\theta - 1)} + e^{-i(\theta - 1)} + e^{-i(\theta+1)}$$
$$= 2 cosh(i(\theta + 1)) + 2 cosh(i(\theta - 1))$$
$$ 2 cosh(i\theta)cosh(i) = cosh(i(\theta + 1)) + cosh(i(\theta - 1))$$
Since $cos(\theta) = cosh(i\theta)$,

$$cos(\theta + 1) = 2 cos(1) cos(\theta) + (-1) cos(\theta - 1)$$
\pagebreak

Equivalently,

$$cos(u) = 2 cos(1) cos(u-1) + (-1) cos(u - 2)$$

We can write the angle of any entry in A as $u = r + 10(j-1)$, where $r$ is the row and $j$ is the column. Therefore, we can subtract $1$ from $u$ to get the row above it as long as $r > 1$.\\

Let $u = 3 + 10(j-1)$ where $j = 1, 2, \dots, 10$. Then $$A(3, j) = cos(u)$$ We can get all of Row 2 by $$A(2, j) = cos(u-1)$$ and all of Row 1 by $$A(1, j) = cos(u-2)$$. 

We can write $cos(u)$ as a linear combination of $cos(u-1)$ and $cos(u-2)$, where those are entries of Row 2 and Row 1, respectively. More generally, let $A(r)$ be the $r^{th}$ row of $A$, $r \geq 3$, 

$$A(r) = (2cos(1))A(r-1) + (-1)A(r-2)$$

Therefore, Row 3 is a linear combination of Rows 1 and 2 and $\det A = 0$, which is an integer.
\end{document}
