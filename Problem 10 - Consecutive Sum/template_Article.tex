\documentclass[]{article}
\usepackage{amsmath}
%opening
\title{Problem 10}
\author{Evan Burton, ID: 010945129,  Undergraduate}

\begin{document}

\maketitle

\section{Problem}
There is a list of 2015 consecutive positive integers such that the sum of the squares of the first 1008 integers is equal to the sum of the squares of the last 1007 integers.  Find the first integer in this list. 

\section{Solution}
Proposition: The quadratic $\displaystyle{\sum_{i=0}^{n}(x+i)^2 - \sum_{i=0}^{n-1}(x+i+n+1)^2} = 0$ has two real solutions for $n \geq 1$ and they are $-n$ and $n(2n+1)$.\\

Proof.\\
$$\displaystyle{\sum_{i=0}^{n}(x+i)^2} = (n+1)x^2+(n^2+n)x+\frac{1}{6}(n)(n+1)(2n+1)$$

$$\displaystyle{\sum_{i=0}^{n-1}(x+i+n+1)^2} = nx^2+(3n^2+n)x + \frac{1}{6}(n^2-n)(2n-1) + 2n^3 + 2n^2$$

$$\displaystyle{\sum_{i=0}^{n}(x+i)^2 - \sum_{i=0}^{n-1}(x+i+n+1)^2} = 
x^2 - 2n^2x-n^2-2n^3 = 0
$$

Which has roots obtained by:

$$x = \frac{2n^2\pm \sqrt{4n^4+4(n^2+2n^3)}}{2}$$
$$= {n^2\pm \sqrt{n^4+n^2+2n^3}}$$
$$= {n^2 \pm n(n+1)}$$
$$x = -n\text{ and } x = n(2n+1)$$

Then for a list of 2015 positive integers, we have: $2n+1 = 2015$, $n = 1007$. Which has solutions $-1007$ and $1007(2015)$, since the numbers are positive, the first number is $1007(2015) = 2,029,105$.
\end{document}
